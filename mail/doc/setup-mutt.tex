\documentclass{article}
\usepackage{listings}
\usepackage{color}
\usepackage{hyperref}
 
\definecolor{dkgreen}{rgb}{0,0.6,0}
\definecolor{gray}{rgb}{0.5,0.5,0.5}
\definecolor{mauve}{rgb}{0.58,0,0.82}
 
\lstset{ %
  language=bash,                % the language of the code
  basicstyle=\ttfamily,           % the size of the fonts that are used for the code
  %numbers=left,                   % where to put the line-numbers
  numberstyle=\footnotesize,          % the size of the fonts that are used for the line-numbers
  stepnumber=1,                   % the step between two line-numbers. If it's 1, each line 
                                  % will be numbered
  numbersep=5pt,                  % how far the line-numbers are from the code
  backgroundcolor=\color{white},      % choose the background color. You must add \usepackage{color}
  showspaces=false,               % show spaces adding particular underscores
  showstringspaces=false,         % underline spaces within strings
  showtabs=false,                 % show tabs within strings adding particular underscores
  frame=bt,                   % adds a frame around the code
  tabsize=2,                      % sets default tabsize to 2 spaces
  captionpos=b,                   % sets the caption-position to bottom
  breaklines=true,                % sets automatic line breaking
  breakatwhitespace=false,        % sets if automatic breaks should only happen at whitespace
  title=\lstname,                   % show the filename of files included with \lstinputlisting;
                                  % also try caption instead of title
  numberstyle=\tiny\color{gray},        % line number style
  keywordstyle=\color{blue},          % keyword style
  commentstyle=\color{dkgreen},       % comment style
  stringstyle=\color{mauve},         % string literal style
  escapeinside={\#*}{*)},            % if you want to add a comment within your code
  morekeywords={*,...}               % if you want to add more keywords to the set
}

%\lstdefinelanguage{bash} {
  %morekeywords={sudo, apt-get, install}
%}

\title{Set Up The Mutt Client}
\author{Conghui He}
\begin{document}
\maketitle

\section{Why Use Mutt}
More and more GUI MUA available today such as web client, MS Outlook, 
Thunderbird and so on, why do you choose to use the console-base and old 
mail client, mutt? The answer is quite simple:
\begin{quote}
  All mail clients suck. This one just sucks less.\footnote{me, circa 1995}
\end{quote}


I am Linux fan and crazy about the efficiency of using Linux and the tools 
that the Linux hackers use. Mutt is one of them, and it is suitable for 
those who has to handle bundles of emails every day. I am not a chief leader 
nor a manager, but rather a newbie to the computer science world subscribing 
several mailing list, thus it is one of the daily work to read the 
nearly 50 pieces of emails a day post by others and myself for a typical 
discussion. A good and easy to customize MUA is assert to me. Here, Mutt is 
the one.

\section{Fire Up Mutt}
This section explains the instructions to install the \textbf{mutt} into 
your own operating system, and provides some configuration tips here. As far 
as I am using Debain like distribution, the following command has to be 
changed if you are using other kinds of distribution.

\subsection{My Mutt Suite}
Some would point out that mutt itself could really technically do most of 
things including sending mails and retrieving mails from the server. This 
absolutely works, but I like to have a division of labor of sort, splitting 
the fetching, sorting, sending, and reading between different programs. 
After all, it is the \emph{Unix Philosophy}.
\begin{quote}
  Write programs that do one thing and do ti well. Write programs to work 
  together.\footnote{Doug Mcllroy}
\end{quote}

I have work with gmail for several years and now switch to mutt. So the 
example here is to shows how to set up gmail in mutt. It is similar to set 
up other email account by only make little modification.

The mutt suite I use includes:
\begin{itemize}
    \item offlineimap
    \item imapfilter
    \item mutt
    \item msmtp
    \item archievemail
    \item vim
    \item goobook
\end{itemize}

The following subsections will detail explains each part of the tools and 
the how to make the configuration along with them.


\subsection{Synchronizing mail with offlineimap}
There are two ways to fetch the emails from the remote mail server by using 
the POP and IMAP protocols. 


POP protocol is suitable for conventional email server. With this protocol, 
you will download all the mails from the remote mail server by indicating 
whether to delete the email you download from the server or not. If you 
don't delete them, the Inbox in the remote mail server will populated with 
hundreds of mails. But I you delete them, and you delete the local mails 
accidentally, referring to the old mails will be nearly impossible. What's 
more, It is note suitable for todays more advanced emails features, such 
labeling and subdirectories.

IMAP is born to deal with such difficulties. It can directly connect to the 
remote mail server and performs the manipulations there, such as move an 
email from one folder to another, which is similar to label the email. 
What's more, you don't have to download all the mails from the server as 
you can and have to view the emails online. This can save lots of space on 
disk, however, it decrease the efficiency as it depends on the speeds of 
your network. On the other hand, you can not read your old mails if you are 
offline.

Here come the offline IMAP. Offline Imap uses the IMAP protocol to 
synchronize a local copy of all your emails messages for multiple accounts 
in a maildir format. Having your email be local means that reading is fast, 
since the emails have already download to your machine. Any changes you 
make, such as deleting, changing flags (important, read, unread, starred, 
\ldots), or moving the email to a different folder are performed on the 
local copy, which is periodically synchronized with the remote copy. That is 
what I need to fetch the gmail.


\subsection{Sorting mail with imapfilter}
If you are experienced gmail user, you must be familiar with the filtering 
ability of the gmail. You can specify different kinds of criteria for 
specific mails to be performed with specific operation. For example, you can 
move the email from \textsf{mutt-users@mutt.org} to a specific folder and 
mark as starred, or you can delete the mail containing \textsf{violence} key 
words and so on. 


Gmail has a powerful ability to make it. So, if you are using gmail, you can 
either set the filter on the gmail remote server side, or make the 
configuration locally. But If you have mail accounts that the remote server 
side doesn't provide this function, you have to do it yourself.


\subsection{Reading mail with mutt}
Supposing that you have synchronized your mails from the server and 
performed the necessary filtering, it's time to set up mutt and to read it. 
Configure your mutt client is an art as configuring vim. There are one 
hundred .muttrc if there are one hundred users there. The details of the 
configuration will shown later and I just what kinds of thing should be 
configured here. 

\begin{itemize}
  \item \emph{account information}. The account information including the 
    name of the user and password of the user should be given.
  \item \emph{color}. Color is one of the attractive reason why someone 
    switch to mutt. The quotations, header, footer, signature, and so on can 
    have their own color defined by yourself.
  \item \emph{encryption}. You'd better not to store your password in plain 
    text, encryption here is necessary.
  \item \emph{mailing list}. You can what your mailing lists are and some of 
  which are subscribed.
  \item \emph{address book}. The address book could store the contacts for 
    you.
\end{itemize}

\subsection{Sending mail with msmtp}
Msmtp is a very simple SMTP client. It contacts the SMTP with your account 
credentials when you send mail and then passes the sent email to the server.

\subsection{Saving old emails with archivemail}
If you have thousands of mails, you'd better archive them for better storing 
and organization. This is quite flexible since you can specify the different 
criteria to select the mails.

\subsection{Synchronize your contacts with goobook}
If you are using gmail, I am sure the gmail contacts mean a lot to you. It 
is necessary to synchronize the gmail contacts with the mutt address book.

\section{Detail Configurations}
Now presents the detail configurations step by step.

\subsection{Preparation}
In Debain-like Operating system, some package should be installed for 
further use. 

%\lstinputlisting[language=bash]{preparation}
\lstset{caption={Install necessary packages}}
\begin{lstlisting}
  sudo apt-get install mutt -y
  sudo apt-get install urlview -y
  sudo apt-get install muttprint -y
  sudo apt-get install offlineimap -y
  sudo apt-get install imapfilter -y
  sudo apt-get install msmtp-mta -y
  sudo apt-get install archivemail -y
\end{lstlisting}

With all the necessary required packages installed, we next turn to fire up 
the folder and files.

It is easier to keep everything together under one directory for git to keep 
track of and for scripts for installation and configuration. Different 
programs have their own configuration file with specific name and in 
specific location, however, we can accommodate to it by make some symbolic 
files.

\lstset{caption={Building folder hierarchy}}
\begin{lstlisting}
  # every configuration is populated in ~/.mail_configs
  mkdir ~/.mail_configs

  # main folder for mutt configuration
  mkdir ~/.mail_configs/mutt

  # different filtering criteria stored here.
  mkdir ~/.mail_configs/imapfilter

  # make this directory readable to you only
  chmod -R 700 ~/.mail_configs

  # make another set of folders to hold the local
  # copy for offlineimap
  mkdir ~/.mail # containing all accounts
  # each account in a separate folder
  mkdir ~/.mail/gmail 
  chmod -R 700 ~/.mail

  # make symblinks for default mutt directory
  ln -s ~/.mail_configs/mutt ~/.mutt
\end{lstlisting}

\subsection{Configurate offlineimap}
A 
\href{https://github.com/jgoerzen/offlineimap/blob/master/offlineimap.conf}{fully 
annotated configuration file} is available if you want to see all the bells 
and whistles available. 

The default offlineimap configuration file is named \textsf{~/.offlineimap}. 
As mentioned above, all the configuration files are stored in a single 
directories and the other necessary files are created by \textsf{ln} command.
So, now we save the offlineimap configuration file \textsf{offlineimaprc} in 
\textsf{~/.mail\_configs/offlineimaprc} and make a symbolic link named 
\textsf{~/.offlineimaprc} to it.

\lstset{caption={configuring offlineimap}}
\begin{lstlisting}
  # create the symbolic link for offlineimap configs
  ln -s ~/.mail_configs/offlineimaprc ~/.offlineimaprc
\end{lstlisting}

When you have configurated your own offlineimaprc, then you can test it from 
the terminal to make sure that it's working at this point. If you are trying 
to connect to Gmail, be sure you have \textbf{enable IMAP} in the Gmail 
interface.
\begin{lstlisting}
  # test the offlineimap configuration
  offline -c ~/.offlineimaprc
\end{lstlisting}

\subsection{Configurate imapfilter}
\href{https://github.com/lefcha/imapfilter}{Imapfilter} is another program 
which is able to sort and move messages around on the IMAP server. I use it 
before calling offlineimap to sort my mail into the correct folder (ie, 
messages for a certain class into a given folder).  The configuration file 
can use 
\href{http://en.wikipedia.org/wiki/Regular\_expression\_examples#Examples}{regular 
expressions} to do the sorting, and is extendable using Lua, if you want to 
do so.
Sorting can take place based on who an email is sent from, sent to, title 
contents, body contents, other header information, email size, age of the 
email, email status (recent, unread,…) and many other possibilities in 
combination.

Alternative, using the filter integrated in Gmail is a good approach which can 
reach the similar effect.
\subsection{Configurate encryption}
Encryption is quite important. At least you should store your email password 
in plain text, which is not only secure but also difficult for you to share 
your work with your friends. What's more, Encrypting the email content is also 
important as the content may contains confidential information.

To configurate the encryption, you should first install the required package, 
with the following command.
\lstset{caption={configurate encryption}}
\begin{lstlisting}
  sudo apt-get install python-keyring -y
\end{lstlisting}

\end{document}
